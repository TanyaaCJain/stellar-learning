\chapter{Process Model}
\begin{epigraphs}
  \qitem{If the automobile had followed the same development cycle as the computer, a Rolls-Royce would today cost \$100, get a million miles per gallon, and explode once a year, killing everyone inside.}
        {Robert X. Cringely, InfoWorld magazine}
\end{epigraphs}
\section {What is Agile?}
Software development is a unique process. Various models have been formulated for summarising and comprehending it but each lacked in one or other aspect and hence were not 100\% applicable in every scenario. Then came along the concept of Agile Methodologies. Wikipedia states \emph{Agile Software Development is a group of software development methodologies based on iterative and incremental development, where requirements and solutions evolve through collaboration between self-organizing, cross-functional teams.} 
Manifesto for Agile Software development was published after 17 software developers met at Snowbird, Utah and discussed about lightweight development methods. The manifesto reads as follows:
\begin {center}
  \begin{longtable}[h!]{|p{0.8\textwidth}|}
    \hline
    \textbf{The Agile Manifesto}\\
    \hline
    
    We are uncovering better ways of developing software by doing it and helping others do it. Through this work we have come to value:

    \textbf{Individuals and interactions} over processes and tools \\
    \textbf{Working software} over comprehensive documentation\\
    \textbf{Customer collaboration} over contract negotiation\\
    \textbf{Responding to change} over following a plan\\
    That is, while there is value in the items on the right, we value the items on the left more.\\

    Twelve principles underlie the Agile Manifesto, including:
    \begin{itemize}
    \item Customer satisfaction by rapid delivery of useful software
    \item Welcome changing requirements, even late in development
    \item Working software is delivered frequently (weeks rather than months)
    \item Working software is the principal measure of progress
    \item Sustainable development, able to maintain a constant pace
    \item Close, daily co-operation between business people and developers
    \item Face-to-face conversation is the best form of communication (co-location)
    \item Projects are built around motivated individuals, who should be trusted
    \item Continuous attention to technical excellence and good design
    \item Simplicity
    \item Self-organizing teams
    \item Regular adaptation to changing circumstances
    \end{itemize}\\
    \hline
  \end{longtable}
\end{center}

\section{So was it Agile?}
Since its inception, creating FShaSycApp with python and PyQt was a venture into the unknown and finally the goals were achieved via exploration and experimentation. Hence. the development life cycle model followed was very much like Agile. It cannot be termed to be Agile because there was only one person working on the project while in a true sense, the concepts of Agile include a team. The practices which bring the followed model close to Agile are listed below:
\begin{enumerate}
\item \textbf{Short term goals}: Since it wasn't known how different aspects of the application would be created, creatign the whole application in one go was not the idea at all. At any stage the aims set were those which seemed feasible with the then level of expertise and knowledge.
\item \textbf{Daily standup-meets}: The goals were set on a daily basis in the twice a day stand-up meets at AmiWorks. 
\item \textbf{Interaction with end users}: Since the application was being developed for use in a small organisation like AmiWorks, talking to colleagues and discussing progress and getting feedback can be compared to frequent interaction with the customers.
\item \textbf{Openness to change}: As mentioned above, there was no predefined path or recipe for creating the application, so there was no rigid approach. If a technique taken for a particular problem seemed to be unappropriate, it was discarded and a different approach was tried out. These were instances where benefits of the use of git (refer to section \ref{sec:git}) for source code management were visible.
\end{enumerate}

\section{Feasibility}
Before starting off with a project, an engineer has to ponder on whether there would be a usable  end product or not. This is feasibility. As the development model being followed was Agile, the question of feasibility popped up for every feature and step. This was tackled with research followed by testing of found solutions. Because of extensive documentation available on the development tools and experienced and helpful programmers on the respective IRC channels, feasibility never became a major issue.
