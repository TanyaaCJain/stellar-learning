\chapter{Results and Discussion}
\label{chap:resultsAndDiscussion}
\begin{epigraphs}
  \qitem {The combination of hard work and smart work is efficient work.}
         {Robert Half}
\end{epigraphs}

Inspired by the keynote talk on Bayesian Statistics by Chris Stucchio, Data Scientist at Wingify, attended at the PyDelhi Conference 2017, a conference for Python programming User Groups; I was inspired to take it up as my topic for my term paper. Since Bayesian Statistics is regarded as an intense topic under Statistics and Probabilities, I had the responsibility to first read in depth about the basics under the topic. Understanding the mathematical notations used in the subject, weren't always a cake walk, but was a great learning experience to understand it further. \par
In the first week I started with understanding the Classical,Frequential and Bayesian Statistics on introductory level as a part of Data Science.I analysed why Bayesian statistics proves to be better than the former two methods. I learnt about the need and benefits to study it and know what makes it a trending topic in Data Science. My journey in understanding Bayesian statistics was aided by an online MOOC offered by University of California Santa Cruz and reading articles available on Wikipedia.\par
For week two, I searched in depth the applications of Bayesian statistics and the wide variety of fields they were used in. With some further research and reading documentations and tutorials, I learnt about Bayesian A/B testing. I chose it as my topic as it is an essential task for both a programmer and a designer. \par
In week three, I studied in depth about A/B testing and how Bayesian statistics can be incorporated in it for better and efficient results. I admired the benefits of Bayesian A/B testing and started working on understanding how to carry it out. I realised my need to practice Python programming, especially the SciPy Python library consisting of Numpy and Matplotlib modules. I took this up as an exercise simultaneously. I also delivered a talk on Exploring Numpy meanwhile in a PyDelhi meetup. I formulated the steps to be carried out during A/B testing which are well formulated in chapter \ref{chap:devTools}. This initiated my coding process. \par \newline
\emph{Bayesian A/B testing is all about forming priors and using the hypothesis to collect data. It is observed that as more and more data is collected the posterior probability gets further away from the priors.} \par
\newline
Statistical significance is the probability that the distinction in change rates between a given variety and the pattern is not because of an arbitrary shot. A consequence of an analysis is said to have factual centrality, or be measurably huge, on the off chance that it is likely not caused by chance for a given factual importance level.\par
Your measurable noteworthiness level mirrors your hazard resistance and certainty level. For instance, in the event that you run an A/B testing try different things with a hugeness level of 97\% or the odds of 97:3, this implies on the off chance that you can be 97\% sure that the watched comes about are genuine and not a mistake caused by irregularity. It additionally implies that there is a 3\% chance that you could not be right. \par
Statistical significance is a method for numerically demonstrating that a specific measurement is dependable. When you settle on choices in view of the consequences of tests that you're running, you will need to ensure that a relationship really exists. Online web proprietors, advertisers, and promoters have as of late turned out to be keen on ensuring their a/b test tests (example, transformation rate a/b testing, advertisement duplicate changes, email title changes) get factual centrality before forming a hasty opinion. \par
Statistical significance is most for all intents and purposes utilized as a part of measurable speculation testing. For instance, you need to know regardless of whether changing the shade of a catch on your site from red to green will bring about more individuals tapping on it. This statistical significance helps to understand the significance of one hypothesis than other variant hypothesis. \par Though, it does not tell the magnitude buy which one variant stands superior to the other. After a great deal of study in the various probability distributions, especially the Beta distribution to help define the prior and hence the Posterior probabilities, I realised my need to know about R programming which would ease out my way to perform Monte Carlo simulations. These simulations have helped to validate the findings derived in Bayesian A/B testing in terms of magnitude. \par
\newpage
Talking about accomplishments, the targets and accomplishments on a weekly basis has been summarized in table \ref{tab:schedule}.

\begin{center}
  \begin{longtable} {|c|p{5cm}|p{5cm}|}
    \hline\\
    \textbf{Week} & \textbf{Target} & \textbf{Accomplishment}\\
    \hline
    1& Get an overview on Bayesian Statistics and know other similar statistics' topics. Read about its pros, cons and applications.& Understanding the Classical,Frequential and Bayesian Statistics on introductory level. Examining why Bayesian statistics proves to be better than the former two methods. Learning about what makes it a trending topic in Data Science and hence its need.\\
    \hline
    2& Read practical real world applications of Bayesian Statistics in all industries inclusive of tech and find one to be the basis of my term paper.& Get acquainted with the basics of topics under Statistics and Probabilities. Chose Bayesian A/B testing as my main focus of term paper.\\
    \hline
    3& Study in depth about A/B testing and how Bayesian statistics can be incorporated in it for better and efficient results.& Read about benefits of Bayesian A/B testing and how to carry it out. Practised Python and the SciPy library.\\
    \hline
    4 & See if any other topics are needed to be done for better performance of A/B tester.& Learnt the basics of R programming to carry out Monte Carlo simulations, that aid in validating results of A/B testing.\\
    \hline
    \caption{Weekly log of accomplishments}
  \end{longtable}
  
  \label{tab:schedule}
\end{center}