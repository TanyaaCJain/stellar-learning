\chapter{Introduction}
\begin{epigraphs}
\qitem{Frequentist arguments are more counter-factual in nature, and resemble the type of logic that lawyers use in court. Most of us learn frequentist statistics in entry-level statistics courses. A t-test, where we ask, “Is this variation different from the control?” is a basic building block of this approach."}%
      {Leonid Pekelis (Data Scientist)}
\end{epigraphs}      

Fundamentally, utilizing a Frequentist strategy implies making forecasts on basic certainties of the trial utilizing just information from the present investigation. \par

\section{A/B Testing}
A/B testing (otherwise called split testing or pail testing) is a technique for contrasting two renditions of a page or application against each other to figure out which one performs better. AB testing is basically an investigation where at least two variations of a page appear to clients aimlessly, and measurable examination is utilized to figure out which variety performs better for a given change objective.\par

\subsection{History}
Like most fields, setting a date for the appearance of another strategy is troublesome in view of the persistent advancement of a subject. Where the distinction could be characterized is the point at which the change was produced using utilizing any accepted data from the populaces to a test performed on the specimens alone. This work was done in 1908 by William Sealy Gosset when he modified the Z-test to build Student's t-test.\par

Google engineers ran their initial A/B test in the year 2000 trying to figure out what the ideal number of results to show on its web index page would be. The principal test was unsuccessful due to glitches that came about because of moderate stacking times. Later A/B testing examination would be more higher in class, yet the establishment and fundamental standards, for the most part, remain, and in 2011, 11 years after Google's initial test, Google has run more than 7,000 diverse A/B tests.\par

\subsection{Why do A/B Testing?}
A/B testing permits people, groups, and organisations to roll out watchful improvements to their client encounters while gathering information on the outcomes. This enables them to develop theories and to learn better why certain components of their encounters affect client conduct. In another way, they can be demonstrated wrong—their sentiment about the best involvement for a given objective can be demonstrated wrong through an A/B test.\par

Something other than noting an erratic inquiry or settling a difference, AB testing can be utilised reliably to consistently enhance a given affair, enhancing a solitary objective like change rate after some time.\par

For example, a B2B innovation organisation might need to enhance their prospective customer quality and volume from crusade points of arrival. Keeping in mind the end goal to accomplish that objective, the group would attempt A/B testing changes to the feature, visual symbolism, frame fields, invitation to take action, and general design of the page.\par

Testing one change at any given moment causes them pinpoint which changes affected their guests' conduct, and which ones did not. After some time, they can consolidate the impact of numerous triumphant changes from trials to exhibit the quantifiable change of the new experience over the old one.\par

\section{Document Structure}
The introductory chapter has laid a more in depth base on the objective of the paper. It introduces the readers to a topic in the field of Statistics and Probabilities which is not only favourable for programmers, but designers as well. The tools used in the development of the A/B tester and its application will be discussed in the upcoming chapter.
Chapter \ref{chap:devTools} lays down the various tools that have aided in the successful development of the A/B tester. It covers the materials including the programming languages, their framework and their modules, as well as the methods in the utilisation of these materials.
The well laid formatting and structure is a result of the utilities provided by the tool, \LaTeX. 